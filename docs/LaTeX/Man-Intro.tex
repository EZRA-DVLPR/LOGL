\section{Introduction}

Hello! This document is intended to be a user-friendly guide to the
software LoGL (Library of Game Lengths). This software is released as
FOSS (Free and Open-Source).
FOSS means that the code for the program is visible to
the public and the resulting program is available for free!

If you are interested in how this program was created, I have created
Documentation that is
\href{https://github.com/EZRA-DVLPR/GameList/blob/main/docs/PDF/Documentation.pdf}{visible
here.}

If at any point, you feel that the document is insufficient, then I
have also created a series of video tutorials on YouTube showcasing
step by step how to use the program, just like in this Manual.
% TODO: Add the link here

\subsection{About}

LoGL is a program that allows users to maintain a library of games
with associated length data from a couple of sources. This program
displays the data in a legible format and allows user a degree of
customization with themes.

\subsection{Ethical Considerations}

LoGL utilizes web scraping in order to obtain game data. This means
that the program will try to connect to websites pretending to be a
normal user, then obtain the data from the webpage, and extract the
relevant information. This is a point for ethical consideration, as
some may feel that this is negatively affecting the visited
websites (e.g. \href{https://howlongtobeat.com/}{https://howlongtobeat.com/}).

If you feel that this is unethical, then I highly encourage you to
not use the program. This program is not illegal, nor was it created
with bad intentions.

\subsection{Why make this Program?}

I often find myself wanting to knock another game off of the gaming
backlog, but see the massive list and feel unsure as to which to play
next. One key factor that weighs into whether I will select a
particular game will be the time it takes to complete it. For
example, to complete a game in 10 hours would be more desirable for a
single week experience vs an 80+ hour game which would probably last
several weeks to months. This information would make selecting a game
to remove from the backlog much easier.

Furthermore, I don't want to always have to go to
\href{https://howlongtobeat.com/}{howlongtobeat} (HLTB) and search for my game,
then compare it to others that I then have to search as well. This
process could take several minutes at a time, depending on how many
games I want to compare. I want a single point of information of
games and their times to complete, so that I can hold the information
that caters to me as an individual based on my owned games.

TLDR;
I want an individualized set of data that contains relevant
information to the games that I have/own with information on the
relative time to complete the game. This would help me clear
the backlog a lot easier.

My solution was to create a piece of software that is platform
independent that allows you to keep track of how long it takes to
complete a game in your library. This should include any game from
any library as well as user input games. The data should be retrieved
from HLTB and other websites that may offer the same information.
This program must have the following qualities:

\begin{itemize}
	\item Able to export to a variety of other programs should the user
		wish to share, or view the data in their preferred program such as
		excel, obsidian, etc.
	\item Be updated with more recent data if user desires (online
		lookup for information from sources)
	\item Customizable with different colored themes with Light/Dark
		themes as a base
	\item Easy to use
	\item Favorite games to be sorted towards the top always
	\item GUI (to see changes in real time. also user-friendly)
	\item Have a guide on how to use the software application
	\item Integrate into as many game services as I was able to do.
		Currently, that is 4 with: GOG, Steam, Epic, and PSN.
	\item Lightweight, efficient and fast data storage, modification, and retrieval
	\item Platform agnostic for desktop (macOS, Linux, and Windows)
\end{itemize}

\subsection{Requirements}

This program requires an internet connect for obtaining data from the
internet which includes processes such as updating, and searching for
game data. Furthermore, it requires the user to download the
\href{https://www.google.com/chrome/index.html}{Google Chrome browser.}
